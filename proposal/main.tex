\documentclass[journal,onecolumn, draftclsnofoot, 12pt]{IEEEtran}
\usepackage[margin=1in]{geometry}
\usepackage[utf8]{inputenc}
\usepackage{graphicx}
\usepackage{times}
\usepackage{amsmath,amsfonts,amssymb,amsthm,commath,dsfont,enumitem}
\usepackage[colorlinks,urlcolor=blue,citecolor=blue]{hyperref}
\usepackage{xcolor}

\usepackage{listings}
\usepackage{todonotes}
\usepackage[T1]{fontenc}
\lstset{language=Python,
basicstyle=\fontfamily{fvm}\selectfont\footnotesize}

% macros adapted from matus & djhsu
\def\ddefloop#1{\ifx\ddefloop#1\else\ddef{#1}\expandafter\ddefloop\fi}
\def\ddef#1{\expandafter\def\csname bb#1\endcsname{\ensuremath{\mathbb{#1}}}}
\ddefloop ABCDEFGHIJKLMNOPQRSTUVWXYZ\ddefloop
\def\ddef#1{\expandafter\def\csname b#1\endcsname{\ensuremath{\mathbf{#1}}}}
\ddefloop ABCDEFGHIJKLMNOPQRSTUVWXYZ\ddefloop
\def\ddef#1{\expandafter\def\csname c#1\endcsname{\ensuremath{\mathcal{#1}}}}
\ddefloop ABCDEFGHIJKLMNOPQRSTUVWXYZ\ddefloop

\DeclareMathOperator*{\argmin}{arg\,min}
\DeclareMathOperator*{\argmax}{arg\,max}
\DeclareMathOperator*{\softmax}{softmax}

\newenvironment{Q}{\item}{\phantom{s}}
\newenvironment{Solution}{\color{blue}\begin{enumerate}}{\end{enumerate}}
%\newcommand{\todo}[1]{\textbf{\color{red} [TODO] #1}}

\title{ECE420 Project Proposal\\ Your project title }
\author{Ethan Zhou, Eric Tang \\
\{yz69, leweit2\} @ illinois.edu}

\begin{document}
\maketitle
\section{Introduction}
Inspired by "South Park" season 18 episode 4, where Randy, the protagonist's dad, assumes the identity of famous female singer Lorde using auto-tune. The goal for this project is to implement an auto-tune app that takes in any songs with vocals, and audio of the user singing, separates the vocals from the songs, and remaps the user's voice into the song with temporal and pitch correction applied. The Repeating Pattern Extraction Technique (REPET) will be used for music/voice separation, and TD-PSOLA will be used for voice synthesis

\section{Overview of the algorithm}
\begin{itemize}
    \item Explain the algorithm to be implemented. Please try to explain the papers in your languages.
    \item Include citation of sources. For example, you can cite the first audio processing paper \cite{wang2003industrial}. Use ref.bib file to manage the citation (all the recommended papers are already included.)  
    \item You can screenshot the figures and diagrams from the paper. Make sure the resolution of the copied source is high enough in your proposal.
    \item You CANNOT screenshot the equations in the paper. Please type the equations. 
\end{itemize}


This is an example of inserting an equation. 
\begin{equation} \label{eq:first}
    \sum_{i=1}^N  \sum_{d=1}^D (\widetilde{s}_{i, T+d} - {s}_{i, T+d})^2
\end{equation}
The equation Eq.~\ref{eq:first} can be referenced. 


This is an example of inserting a figure. 
\begin{figure}[h]
\begin{center}
%\includegraphics[width=0.5\textwidth]{banner.png}\\
\caption{ This is the banner of ECE 420 webpage.  } 
\label{fig:banner}
\end{center}
\end{figure}
The Figure~\ref{fig:banner} can be referenced. 


This is an example of inserting a table.
\begin{table}[h]
\small
    \centering
    \begin{tabular}{|p{1.5in}|p{1.5in}|}
        \hline
        \textbf{Contents} & \textbf{Number} \\
         \hline
        task1 &  1\\
        \hline
        task2 &  2 \\
         \hline
         
    \end{tabular}
    \vspace{0.1in}
    \caption{Tasks and allocations}
    \label{tab:table}
\end{table}
The Table~\ref{tab:table} can be referenced. 


\section{Plan for testing and validation}
\begin{itemize}
    \item How will you demonstrate it works?
    \item What inputs will you be using?  Are those pre-existing or do they need to be generated?
    \item What are the outputs?  Is it objective or subjective?  Can you collect metrics?
    \item Start with ‘easy’ test cases, work your way up to more complex. 
\end{itemize}

\section{Contribution}
\begin{itemize}
    \item Author 1: task A, B, C
    \item Author 2: task 1, 2, 3
\end{itemize}



 
\bibliographystyle{IEEEtran}
\bibliography{ref}

\end{document}
